
\documentclass[12pt]{article}

\usepackage{sbc-template}

\usepackage{graphicx,url}

\usepackage[utf8]{inputenc}  

\usepackage[T1]{fontenc}

\usepackage[brazil]{babel}

\usepackage[utf8]{inputenc}

\sloppy
\begin{document} 

\title{Compartilhamento do conhecimento, chave para evolução}

\author{Natanael Bezerra de Oliveira}

\address{Instituto de Computação -- Universidade Federal do Amazonas
  (UFAM)\\
  Av. Gen. Rodrigo Octávio, 6200, Setor Norte, Coroado - 69080-900 - Manaus - AM
  \email{\{natanael.oliveira\}@icomp.ufam.edu.br}
}


\maketitle

A sociedade humana, desde sua origem até os dias atuais, evoluiu a partir do compartilhamento do conhecimento. E ciência aberta e software livre tem em sua origem o compartilhamento do conhecimento, esse texto se limita a falar o que é ciência aberta e software livre, sua importância e qual o papel da computação na ciência aberta. 

Não há um consenso em relação a ciência aberta, é um termo constantemente utilizado para denotar políticas, iniciativas e ações para disseminar e compartilhar o conhecimento \cite{clau_22_2}, em que os dados devem ser acessíveis, interoperáveis e reusáveis, incluindo também a participação de não cientistas \cite{clau_22}.  Ao falar de ciência aberta não há como deixar de mencionar software livre, esse é o que que vem com a permissão para qualquer um usar, copiar e distribuir, tanto da versão original do software como versões modificadas, de forma gratuita ou cobrando uma taxa \cite{Revolucao}.

Imagine um padeiro, que produz um pão maravilhoso, entretanto é um pão simples e você quer uma pão doce. Caso o padeiro te dê a receita do pão poderá seguir os mesmos passos modificando apenas o necessário para fazer o seu pão doce, mas, se o padeiro se recusar a disponibilizar a receita, terá que descobrir a receita, tarefa que levará muito mais tempo que apenas modificar uma parte. Assim, com a analogia apresentada podemos entender a importância da ciência aberta. Por exemplo, ao compartilhar um artigo em que é disponibilizado os dados adquiridos, outros pesquisadores podem utilizá-los para validar o trabalho original, criar novas versões ou até mesmo corrigir possíveis erros. Desde os primórdios da computação a liberdade de acesso ao código vem sendo motor propulsor e facilitador de novas descobertas \cite{Revolucao}. Por isso, respeitar a liberdade não é levar o outro até onde ele quer, é somente não impedir a passagem dele pelo caminho que leva até onde ele quer chegar \cite{Revolucao}.

E se para fazer o pão o padeiro utilizar um forno especifico? Forno de difícil acesso ou um ingrediente muito complexo de se utilizar? Então não conseguiria criar o pão, mesmo tendo a receita em mãos. Na ciência aberta o mesmo problema surge, por mais que tenha todos os dados, nem sempre será fácil alcançar os mesmo resultados. É nesse cenário que entra a Computação, com a criação de diversas ferramentas que possibilitam que recriar o "pão", podemos destacar o scikit-learn, tensorFlow e pyTorch, ferramenta que possibilitam, entre outras coisa, análise de dados, o qual utilizam inteligência artificial e aprendizado de maquina (algoritmo permite o "computador" aprender com seus erros), ainda permitindo ser modificado conforme a necessidade. 

O software livre veio para ser tornar um forte aliado da ciência aberta, a partir do software livre é possível adaptar ferramentas existentes para atender as suas necessidades e a ciência aberta disseminar o conhecimento, utilizando ferramentas para recriar seus resultados. 
Podemos utilizar uma das leis da física para explicar a importância da ciência aberta e do software livre, “Na Natureza, nada se cria, nada se perde, tudo se transforma”, ninguém adquire o conhecimento do nada, é necessário que previamente alguém compartilhe o conhecimento, para só então esse conhecimento evoluí para outra coisa, então nada mais justo que devolver à sociedade o conhecimento adquirido, para que outros possam evoluir também.

\bibliographystyle{sbc}
\bibliography{sbc-template}

\end{document}